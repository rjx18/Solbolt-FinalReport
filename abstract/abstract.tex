\begin{abstract}

With the recent boom in interest for DeFi platforms and NFTs on
smart contract enabled blockchains such as Ethereum, the cost of performing transactions
in terms of gas prices has increased tremendously. In order to make these transactions
more accessible for general usage, there is a clear need not only for the better
scaling of blockchains, but also for developers to create more gas efficient 
applications. However, most of the developer tools that exist today mainly focus on
security verification rather than gas analysis for Solidity smart contracts.
As such, the main goal of this project is to create a powerful yet user-friendly
tool, that generates detailed gas estimates and detects potential issues,
so as to help developers more easily pinpoint which parts of their contracts can be
refactored to save gas.

The tool we created, \textcolor{NavyBlue}{\textsc{Solbolt}}, makes use 
of symbolic execution by extending the Mythril engine, via its 
plugin and hooks system. \textcolor{NavyBlue}{\textsc{Solbolt}} is able to:
(i) generate a detailed code mapping from the compiled EVM bytecode
to the original Solidity bytecode, via a simple user interface built
using React Typescript and inspired by the popular GCC compiler explorer
tool Godbolt, and a backend running the Solidity compiler and Mythril engine, 
served using Flask and a Celery task queue; (ii) estimate the gas usage for each code mapping, by 
using a custom gas meter plugin and gas hook that we extended the original Mythril
engine with, which stores the total gas used by each code section for each symbolic transaction; 
(iii) estimate the gas usage for each function and each loop iteration, by 
developing a function tracker plugin which made use of the EVM dispatcher routine to
infer the function that was taken by each symbolic execution path,
as well as a loop gas meter plugin which stores the current executional trace
and is able to detect if the trace is currently in a loop; (iv) allow users to directly import verified contracts
from the Etherscan API; and (iv) automatically detect possible gas
inefficient patterns and suggest how they can be fixed, namely the detection of
storage mutations within a loop.

\textcolor{NavyBlue}{\textsc{Solbolt}} is then evaluated
over 1,392 smart contracts, by comparing the estimated gas costs
for each function to the actual transaction gas cost. The median estimated gas
cost was found to be 129.3\% of the concrete gas costs, while the median
coverage was 99.1\%. These results are then compared to other previous studies,
and were found to be comparable with the current state-of-the-art. Finally, 
a case study was performed to determine how a developer might use the tool
to optimise a smart contract.

\end{abstract}