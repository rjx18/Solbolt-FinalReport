\chapter{Conclusion}
\label{chap:conclusion}

As the DeFi and NFT space continues to grow in the coming years, concerns over high gas prices
required to perform any transactions within this space will likely continue. While scaling
decentralised blockchains such as Ethereum remains an open problem being worked on, developers
today can help partially address this by writing better, more gas-efficient smart contracts.

As such, \textcolor{NavyBlue}{\textsc{Solbolt}} is designed particularly for this purpose, to help developers better pinpoint
exactly which parts of their code is using up the most amount of gas, and to more easily identify
gas-inefficient patterns.

\section{Summary of Achievements}

Currently, the majority of Solidity analysis tools available on the market are focused on 
security verification of smart contracts, and often neglect another important part of the
development pipeline --- gas optimisation.
With \textcolor{NavyBlue}{\textsc{Solbolt}}, the compiler-explorer and gas analysis tool we have built, we believe that
we have successfully achieved our objective of creating a tool that helps developers 
better understand the compiled bytecode and gas profile of their contract, and addressed some of the market gaps within this part of the Solidity
development ecosystem, as summarised:

\begin{enumerate}
  \item \textbf{Intuitive and powerful user interface of compiled EVM bytecode and their source maps}
  
    We have successfully built a simple-to-use and intuitive interface for
    viewing the compiled EVM bytecode and their source maps, inspired
    by the tool Godbolt. The frontend is built using React Typescript and served using
    AWS Amplify, while the backend made use of a Celery task queue for the dispatching of
    jobs, as well as Flask for serving the API endpoints. 
    The hover function makes it easy to identify which code
    sections relate to which EVM bytecodes, and the Etherscan API integration
    allows for easy analysis of any verified Ethereum smart contracts. 
    In addition, the tool allows for the compilation of multi-file, multi-contract
    source codes, from a wide range of supported Solidity compilers. 
    Finally, the customisable compiler settings also exposes a detailed
    range of optimisation options.

    To date,
    we know of no other tool that provides a similar user experience when
    analysing the EVM bytecode. We believe that this function can be very
    useful for researchers and developers, who are looking to more closely
    examine their deployed bytecode, or wish to tweak some of the more obscure
    optimisation options.
    
  \item \textbf{Accurate and detailed analysis of gas consumption}
  
    Our extended Mythril symbolic execution engine can accurately estimate the gas consumption of
    most contracts within reasonable margins, given sufficient coverage. This can be
    performed at the code section level, using the gas meter and gas hooks which we have 
    built specifically for this task. Gas estimation can also be done at the functional level
    and the per loop iteration level, by 
    developing a function tracker plugin which made use of the EVM dispatcher routine to
    infer the function that was taken by each symbolic execution path,
    as well as a loop gas meter plugin which stores the current executional trace
    and is able to detect if we are currently in a loop. These features provide detailed insight about the gas profile
    of the smart contract and where possible savings may be, which to our knowledge,
    no other tool is able to generate with the same level of accuracy and fineness.

    The project also proves the efficiency of symbolic execution for not just
    security analysis, but also gas estimation.

  \item \textbf{Automatic detection of potential issues for further gas savings}
  
    \textcolor{NavyBlue}{\textsc{Solbolt}} is also able to automatically detect common gas-inefficient patterns,
    such as modifying storage variables within a loop, to allow developers to 
    quickly pinpoint where potential gas savings can be achieved. This has shown
    to be useful and accurate in the case study we have examined.
    
\end{enumerate}

\section{Limitations}
While we believe that \textcolor{NavyBlue}{\textsc{Solbolt}} is ready for general development use, we also acknowledge some of
the limitations that \textcolor{NavyBlue}{\textsc{Solbolt}} currently has. First, when compared to some state-of-that-art
gas estimation tools such as \textsc{Gastap}, \textcolor{NavyBlue}{\textsc{Solbolt}} still is not able to guarantee a worst-case gas bound,
and is unable to generate paramteric bounds depending on function arguments. \textcolor{NavyBlue}{\textsc{Solbolt}} may also at times
fail to work when using advanced features such as manual linking of libraries, or the use of oracles.

In addition, while the original scope of the work included compilation and mapping of the
Yul intermediate representation, this has been sidelined by the creation of more detailed
plugins for analysis on the EVM bytecode instead. However, we believe that once Yul matures
from its experimental state, such a feature could be very useful in the future. This is especially so
given the planned usage of EWASM over EVM for the future release of Ethereum, where Yul will continue
to act as an intermediate language.

\section{Future work}
\textcolor{NavyBlue}{\textsc{Solbolt}} is still a work in progress and improvements to the user interface and the symbolic engine
can definitely still be made. Some possible ideas that come to mind are laid out as follows:

\begin{enumerate}
  \item \textbf{Support for Yul IR code mappings} The code mappings for the Yul IR can be added to the
  \textcolor{NavyBlue}{\textsc{Solbolt}} pipeline, by first storing the Solidity code maps generated for the Yul IR, and then compiling
  the Yul IR into the EVM bytecode and storing the corresponding Yul code maps. The Yul code maps would then
  need to be merged with the Solidity code maps in order to produce a global code mapping. This would however
  only be supported for more recent versions of the Solidity compiler that support Yul compilation.
    
  \item \textbf{Integration with block explorers for other EVM-compatible chains} Next, in addition to 
  the Etherscan API, we would like to possibly integrate other block explorers
  to analyse contracts across all EVM-compatible chains, such as BSCScan for Binance Smart Chain,
  Snowtrace for the Avalanche C-Chain, and so on.

  \item \textbf{More plugins for detection of gas inefficient patterns} Currently, \textcolor{NavyBlue}{\textsc{Solbolt}} is only able to
  detect storage variable access within loops as a gas inefficient pattern. However, many other gas inefficient
  patterns also exist, such as dead code, opaque predicates, potentially fusible loops, and repeated
  computation of the same result \cite{gaschecker} \cite{designpatternsforgasoptimisation}. These can
  possibly be implemented as additional plugins for Mythril, and be used to quickly detect potentialy avoidable
  gas usage in a smart contract.

\end{enumerate}